
\section{Introduction}\label{Section:Introduction}

%We measure a bunch of \ak jet spectra in \ee collisions using ALEPH data!

Jets, a collimated spread of particles originated from a fast-moving quark or gluon, are some of the most useful tools for studing the Quantum Chromodynamics (QCD) and searching for new physics in high energy colliders beyond the Standard Model (SM). Since the end of the Large Electron Positron Collider (LEP) operation, significant progress has been made for the jet definitions, jet algorithms, and jet substructure observables~\cite{Larkoski:2017jix} However, those techniques, which were widely deplored in the data analyses of the proton-proton~\cite{Kogler:2018hem} and heavy-ion collisions~\cite{Cao:2020wlm}, are not yet used in the most elementary $e^+e^-$ collision system. Phenomenological models,h as \textsc{pythia}~\cite{Sjostrand:2000wi}, \textsc{sherpa}~\cite{Gleisberg:2008ta} and \textsc{herwig}~\cite{Reichelt:2017hts}, are  tuned with hadronic event shape observables and were employed to predict the jet spectra and jet substructures at more complicated hadron-hadron collision environments.

Studies of jets in $e^+e^-$ using identical algorithms as those were used in high-energy hadron colliders such as LHC and RHIC are of great interest.
Unlike hadron-hadron collisions, electron-positron ($e^+e^-$) annihilation does not have beam remnants, gluonic initial state radiations, or the complications of parton distribution functions.
Therefore, electron-position annihilation data provide the cleanest test for QCD and phenomenological models that are tuned with hadronic event shapes.
Moreover, fully reconstructed jets provide us an opportunity to inspect the quark and gluon fragmentation in great detail on a shower-by-shower basis.
Finally, studies of jet substructure and their comparison to modern event generators are of great interest since jet substructure observables are novel tools for jet flavor identification, electroweak boson/top tagging, and studies of the Quark-Gluon Plasma properties at hadron colliders~\cite{Andrews:2018jcm}. 

In this analysis note, the first measurement of anti-$k_{T}$ jet~\cite{Cacciari:2008gp} momentum spectrum, jet splitting functions, and subjet opening angle spectrum in hadronic $Z$ boson decays are presented. There are two types of observables presented. The inclusive observables include all the reconstructed jet above the jet energy threshold and inside a defined acceptance. It is sensitive to higher-order corrections of jet spectra in the language of perturbative QCD since it includes very low momentum jets in the events which are typically associated with soft gluon radiations or random combinatorial jets with hadrons from different partons. On the other hand, the leading dijet observables consider only the leading and subleading jet in the event. This type of observables focuses more on the dominant energy flow and is less sensitive to soft radiations.

The analysis note is organized in the following way: The data and Monte Carlo samples are documented in Section~\ref{Section:Samples}. Hadronic event selection and background rejection criteria are described in Section~\ref{Section:Selection}. Jet reconstruction and calibration procedures are documented in Section~\ref{Section:JetReconstruction} and \ref{Section:JetCalibration}. The simulated jet resolution and data-driven correction to the resolution function are documented in Section~\ref{Section:JetResolution}. Leading jet selection criteria are described in Secion~\ref{Section:LeadingJet}. Resolution unfolding, systematics and the results are summarized in Section~\ref{Section:Unfolding},~\ref{Section:Systematics} and \ref{Section:Result}.


\clearpage
